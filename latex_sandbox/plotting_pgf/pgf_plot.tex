\documentclass{article}
\usepackage{pgfplots}
\usepackage{filecontents}
\usepackage{csvsimple} % lets you import a csv file to make, say, a table
\usepackage{xcolor} % needed to control colors better
\usepackage{tikz} % needed for resize box

% want just a few simple things to be proven here: 
% plot from csv
% control formatting like...
%   grid - done
%   legend - done
%   markers
%   linestyles
%   text size
%   axis placement
%   figure dimensions
%   labels - done
%   optional: perhaps random text





\begin{document}
kjgnote: how to compile this: \texttt{pdflatex pgf\_plot.tex}

\csvreader[head to column names]{dat_multiCols.csv}{}% use head of csv as column names

% First plot, covering the basics
%\definecolor
\begin{figure}[h]
    \centering
    \begin{tikzpicture}
    \begin{axis}[
        width=\textwidth,
        height=6cm,
        xlabel=X Axis,
        ylabel={Y Axis}, % it's ok to have comma on last item, and brackets help
        title=Title of Plot,
        legend style={at={(0.25,0.9)}}, % can also include "anchor", and "at" uses (x%,y%)
        grid=major, % can also be "both", and perhaps "minor"
        ] % figure configuration
        \addplot+[
            style={densely dashed},
            line width = 4,
            color = red,
            mark options ={style=solid,scale=2,mark=square,color=black,line width=2} % marker needs to be reset to "solid"
        ] % individual curve configuration
            table [x=time, y=squared, col sep=comma] {dat_multiCols.csv}; % reference the filename
        \addlegendentry{quadratic}
        % \legend{one,two,three} % can also create legend on own
    \end{axis}
    \end{tikzpicture}
    \caption{Example plot}
\end{figure}







This text only to help get a sense of page width: Lorem ipsum dolor sit amet, consectetur adipiscing elit. 

end
\end{document}